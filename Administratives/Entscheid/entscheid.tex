\documentclass[a4paper,12pt]{scrartcl}
\usepackage[left= 2.5cm,right = 2cm, bottom = 4 cm, top = 2.5cm]{geometry}
\usepackage[utf8x]{inputenc}
\usepackage {graphicx}
\usepackage[ngerman]{babel}
\usepackage{hyperref} %immer als letztes

\begin{document}
\section{Daten Entscheid}
\label{sec:Daten Entscheid}
Für unseren Journey Planer benötigen wir Fahrplaninformationen womit wir unseren Algorithmus füttern. Die Fahrplandaten stehen in zwei verschiedenen Dateiformaten zur Verfügung. 
\newline
\newline
Wir haben uns für das GTFS-Format entschieden.
Laut der Seite\footnote{\url{http://gtfs.geops.ch/doc/}}, die Konvertierung HAFAS-Rohdaten nach GTFS Dokumentation zur Verfügung stellt, erwähnt: "GTFS ist besser strukturiert, einfacher zu benutzen, einfacher zu erweitern und besser dokumentiert als HAFAS-Rohdaten". Zudem wird gesagt das OpenSource Routenfinder wie OTP unterstützen GTFS als einziges Dateiformat für Fahrplandaten.\cite{hrdfintogtfs}
\newline
\newline
Zusätzlich laut der Plattform\footnote{\url{https://opentransportdata.swiss/}} die Daten zur Verfügung stellt warnt zudem noch: "Die HRDF-Datei(en) sind relativ komplex. Ohne Not sollte nicht damit gearbeitet werden."\cite{opentransporthrdf}

Da wir uns für das GTFS-Format entschieden haben ist es auch logisch die dafür zugehörige Erweiterung für Echtzeit zu verwenden (GTFS-RT).

\end{document}