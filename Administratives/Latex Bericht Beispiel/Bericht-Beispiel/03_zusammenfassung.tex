\section*{Abstrakt}
\label{sec:abstrakt}

Die vorliegende Arbeit beschäftigt sich mit Strömungssimulationen um das Laminarflügelprofil des Flugzeugs FK12 Comet S1. Sie geht der Frage nach, wie die Flügelprofile modifiziert werden können, damit sie bessere Langsamflugeigenschaften aufweisen. 
In dieser Arbeit wird als erstes die Modifikation des Flügelprofils mit einer anderen Landeklappe betrachtet, welche beim neuen Flugzeugmodell S2 umgesetzt wurde. Die Ergebnisse zeigen den positiven Einfluss der Landeklappen auf die Langsamflugeigenschaften. Als zweites wird das Aufrüsten des Flügels S1 mit Vortexgeneratoren betrachtet. Vortexgeneratoren sind einfache geometrische Teile, welche auf der Flügeloberfläche platziert werden. Bei dieser Untersuchung wird eine Parameterstudie durchgeführt, um die Vortexgeneratoren optimal auszulegen. In den Ergebnissen ist die Wirkung deutlich zu erkennen und es können auch Rückschlüsse auf die richtige Auslegung gezogen werden. 
Die Strömungsuntersuchungen werden mit ANSYS-CFX durchgeführt; dabei ist speziell das verwendete $\gamma-Re \Theta$ Modell zu erwähnen. Dieses empirische Modell ermöglicht eine Abbildung des Umschlags von der laminaren zur turbulenten Grenzschicht.

\newpage


\section*{Abstract}
\label{sec:abstract}


The present work deals with flow simulations around the laminar airfoil of the airplane FK12 Comet S1. It examines the question, how the airfoil can be modified to improve the slow flight characteristics. 
To start, this work looked at the modification of the wingprofile with another landing flap which is in use with the new airplane model S2. The results show the positive influence of the landing flaps on the slow flight characteristics. Secondly, the upgrade of the wing profile S1 with turbulators was observed. Turbulators are simple geometrical parts that are placed on the airfoil surface. With this investigation a parameter study was performed, to lay out the turbulators optimally. The effect is clearly recognisable in the results and conclusions on the right placement can be drawn. 
The flow investigations are carried out with ANSYS-CFX; in the investigation process the $\gamma-Re \Theta$ model was used. This model allows an exact picture of the turn from the laminar to the turbulent border layer.
