\section[Vortexgenerator auf dem Flügelprofil S1]{Vortexgenerator auf dem Flügelprofil S1 mit Klappenstellung \SI{10}{\degree}}
\label{sec:vortexgenerator_auf_fluegelprofil_s1_mit_klappe_10_grad}
Nun wird untersucht, wie sich ein Vortexgenerator auf dem Flügel des Flugzeuges FK12 Comet S1 auswirkt; das heisst nur ein Teil eines Flügels mit Vortexgenerator wird untersucht. Die Simulationen werden bei einem Flügelanstellwinkel von \SI{15}{\degree}, \SI{17}{\degree} und teils \SI{19}{\degree} durchgeführt, da bei diesen Winkeln der maximale Auftrieb bei den zweidimensionalen Simulationen mit dem Flügelprofil S1 im Kapitel \ref{sec:fluegel_2d} erreicht wurde, was in Abbildung \ref{fig:auftriebsbeiwert_2D} zu erkennen ist. Mit Hilfe der Vortexgeneratoren soll der Auftrieb bei diesen Winkeln erhöht werden und die Strömung um den Flügel soll weiter hinten abreissen. 


\subsection{Modellierung}
\label{subsec:modellierung_fluegel_s1}
Damit aussagekräftige Ergebnisse entstehen, werden die Umgebungsgrösse, die Vernetzung und die Randbedingungen bei allen Simulationen beibehalten, sofern nichts anderes erwähnt wird. Ausserdem wird bei allen folgenden Simulationen transient gerechnet.

und so weiter

