\section[AufgabeBA]{AufgabeBA}
\label{AufgabeBA}


\subsection{AufgabenstellungBA}
\label{AufgabenstellungBA}
\begin{itemize}
	\item Schreiben sie eine Implementation des CSA für den OpenTripPlanner. Dieser muss OTP-Development-Richtklinien konform sein.
	\item Implementieren sie die Schweizer GTFS und OpenStreetMap daten für den OTP, so dass Punkt zu Punkt Verbindungen in der Schweiz berechnet werden können.
	\item Implementieren sie den Schweizer realtime GTFS-Feed, so dass das Programm auf Verspätungen oder Fahrplanänderungen reagieren kann.
	\item Implementieren sie Skalierungsoptimierungen für das Programm, so dass es mit einem landesweiten System bessere Performanz liefert.
	\item Führen sie Performance Tests durch und vergleichen sie das Implementierte System mit dem auf Dijkstra basierenden OTP.
	\item Dokumentieren sie ihre Ergebnisse und schreiben sie einen ausführlichen Bericht.
\end{itemize}

\subsection{Lastenheft}
\label{Lastenheft}
\begin{tabular}{ | p{5cm} | l | l | }
	\hline
	\multicolumn{3}{|c|}{Funktionale Anforderungen} \\
	\hline
	Anforderung & Anforderungsart & Priorisierung \\
	\hline
	Der CSA muss als Berechnungsalgorithmus verwendet werden. & Muss & 1 \\
	\hline
	Es kann eine Verbindung zwischen 2 Stationen zu einem bestimmten Zeitpunkt errechnet werden. & Muss & 2 \\
	\hline
	Fusswege zu Stationen hin können miteinbezogen werden können. & Muss & 3 \\
	\hline
	Das Programm kann auf Verspätungen und Fahrplanänderungen reagieren. & Muss & 4 \\
	\hline
	Mehrere mögliche Verbindungen können angezeigt werden. & Muss & 5 \\
	\hline
	Fusswege zwischen verschiedenen Stationen können miteinbezogen werden. & Soll & 6 \\
	\hline
	Die Angezeigte Route soll den Fahrlinien, z.B. Schienen, folgen. & Kann & 7 \\
	\hline
\end{tabular}
\newline
\newline
\newline
Nicht funktionale Anforderungen
\begin{itemize}
	\item Der Programmcode muss den OTP-Development-Richtlinien entsprechen.
	\item Die Query-Zeit muss weniger als 1 Sekunde betragen.
	\item Die Preprocessing-Zeit muss weniger als 30 Minuten betragen.
	\item Die Query-Zeit soll schneller als die des originalen OTP sein.
\end{itemize}

\subsection{Meilensteine}
\label{Meilensteine}
\begin{enumerate}
	\item OTP Analysiert
	\item OTP mit dem Schweitzer System implementieren
	\item Basis CSA implementiert
	\item CSA Spalierungsoptimierung implementiert
	\item Performancetests durchgeführt
	\item Bericht geschrieben
\end{enumerate}

\subsection{Zeitplan}
\label{Zeitplan}
Der Zeitplan ist im beigelegten Excel-File ersichtlich.