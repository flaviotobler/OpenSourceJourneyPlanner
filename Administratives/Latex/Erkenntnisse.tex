\section[Erkenntnisse]{Erkenntnisse}
\label{sec:erkenntnisse}

\subsection{Daten}
\label{sec:erkenntnisseDaten}
Für unsere Anwendung ist der GTFS Datentyp dem HRDF Datentyp aus mehreren Gründen überlegen.
\begin{enumerate}
	\item Alle zuvor genannten Programme basieren auf dem GTFS-Datentyp. Dies führt
\end{enumerate}



\subsection{Algorithmen}
\label{sec:erkenntnisseAlgorithmen}



\subsection{Programme}
\label{sec:erkenntnisseProgramme}
Die drei Programme Traintickets.to, OTP und R5 wurden in Betracht gezogen. 

R5 basiert zwar auf dem modernen RAPTOR Algorithmus, ist jedoch nicht für unser Projekt geeignet. Der Programmcode steht unter einer MIT-Lizenz. R5 wird von der Firma Conveyal entwickelt. Wenn wir mit unserem Projekt auf das R5 Programm aufbauen, so besteht keine Chance dass unser Code in das Originalprogramm integriert wird.

Die beiden Programme "Traintickets.to" und "OpenTripPlanner" wurden genauer betrachtet. Die Programmierer beider Programme wurden von uns angeschrieben. Wir posteten eine Anfrage in die OTP Developer Mailing. Darin fragten wir ob sie zurzeit an einer implementation des CSA in ihrem Programm arbeiten und wie sie zu der Idee unseres Projektes stehen. Nach nur kurzer Zeit bekahmen wir eine Antwort. Eine integration des CSA ist zurzeit nicht geplant. Projekte von neuen Personen sind gerne gesehen, jedoch wird der Code nur in das Hauptprogramm übernommen, wenn  das OpenSource-Gremium den Code als gut erachtet. Dazu muss der Code den Programmrichtlinen entsprechen und der Strategie des OTP entsprechen. Die strategie des OTP liegt darin, dass sie nicht ausschliesslich auf die schnellste Lösung setzen. Wichtiger ist, dass das Programm in Echtzeit auf Verspätungen und Fahrplanänderungen reagieren kann. Wir haben auch Linus Norton von "Traintickets.to" angeschrieben, konnten jedoch keine Antwort erhalten.