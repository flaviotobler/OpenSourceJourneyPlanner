\section{Ausgangslage}
Um unsere Programm und unser vorgehen zu besser verstehen muessen wir zuerst das von uns verwendete Basisprogramm sowie den von uns verwendeten Basisalgorithms erlaeutern.

\subsection{JourneyPlanning}
Ein JourneyPlanner ist ein Programm, welches den optimalen Weg für eine Reise zwischen zwei oder mehr Orten findet. Im gegensatz zum RoutePlanning oder Routenplanner bezieht sich ein JourneyPlanner nur auf öffentliche Verkehrsmittel und nicht auf Privatfahrzeuge.

\subsection{OpenTripPlanner}

Der OpenTripPlanner, kurz OTP, ist eine auf der Maven-Repository %~\cite{maven_repository} 
aufbauende Multimodale trip planning Software welche anfangs für Städte ausgelegt war, nun aber auch in ersten landesweiten Netzwerken Anwendung findet. Er wurde von einem OpenSource-Kollektiv aus mehr als 100 Personen in acht Jahren entwickelt. %/*~\cite{otp_website}

Der OTP basiert auf dem A*-Algorithmus und verwendet GTFS-Daten und OpenStreetMap Daten in Form einer pbf-Datei. In einem preprocessing Schritt wird der Graph für den Algorithmus erstellt. Dieser kann in einer Datei gespeichert werden oder direkt im RAM des Servers gelagert werden. Selbiges wird für die OpenStreetMap-Daten gemacht. Während dem Betrieb kann der Graph angepasst werden, so dass das Programm auf verspätete Züge reagieren kann. %~\cite{otp_git}

Der OTP erzeugt für jeden Aufruf eine neue Instanz des Graphen. Somit können mehrere Aufrufe parallel behandelt werden.

OTP steht unter einer GNU Lesser General Public License. 


!!! Selbstplagiat



\subsection{Connection Scan Algorithmus}
 

\subsection{General Transit Feed Specification}
%Erklärung was GTFS ist und was dessen aufbau und regeln sind
General Transit Feed Specification (GTFS) ist ein von Google entwickeltes Dateiformat zum Austausch von Öffentlichen Verkehrsdaten sprich Fahrpläne. Die Daten werden von der Platform\footnote{\url{https://opentransportdata.swiss/}} zur Verfügung gestellt. GTFS ist ein statisches Dateiformat und beinhaltet keine Echtzeitdaten, wie Verspätungen, Ausfälle etc. und wird deshalb auch GTFS Static genannt. Die Daten werden in verschiedenen Textfiles zur Verfügung gestellt, welche wiederum viele wichtige Informationen enthält.\newline

\begin{tabular}{|l|c|l|}  \hline
	Dateiname & pflicht? & Definition \\ \hline
	agency.txt & ja & Geschäftsstellen die Daten zur Verfügung stellen \\ \hline
	stops.txt & ja & Haltestellen mit ihrer Position \\ \hline
	routes.txt & ja & Verkehrsverbindungen (Linien) mit den Fahrzeugarten \\ \hline %(zeitunabhängig)
	trips.txt & ja & Fahrten  \\ \hline												%(zeitabhängig)
	stop\_times.txt & ja & Zeiten in der Fahrzeuge Ankommen/Abfahren an Haltestellen \\ \hline
	calendar.txt & ja & Fahrplanveränderungen (Jahreszeiten) \\ \hline
	calendar\_dates & optional & Ausnahmeplan für bestimmtes Datum \\ \hline
	fare\_attributes.txt & optional & Fahrpreise und die Art der Bezahlung \\ \hline
	fare\_rules.txt & optional & Fahrpreisregeln verschiedener Zonen  \\ \hline
	shapes.txt & optional & Beschreibt den Weg eines Fahrzeuges (Darstellung) \\ \hline
	frequencies.txt & optional & Fahrpläne ohne fixe stop Zeiten. \\ \hline
	transfers.txt & optional & Umsteigpunkte verschiedener Routen (Linien) \\ \hline
	feed\_info.txt & optional & Zusätzliche Informationen über den Datensatz \\ \hline	
\end{tabular}

%\cite{gtfsInhalt} %Verweis aus Fachmodul!!
Daten die bisher nicht von der Plattform zur Verfügung gestellt werden: fare\_attributes.txt, fare\_rules.txt, frequencies.txt.


