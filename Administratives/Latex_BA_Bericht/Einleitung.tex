\section{Einleitung}

Ziel der Facharbeit war es, einen OpenSource-\hyperlink{JourneyPlanner}{JourneyPlanner} zu finden, der mit dem grossen Datensatz der Schweizer \hyperlink{GTFS}{GTFS-Daten} funktioniert, oder gegebenenfalls selbst einen JourneyPlanner zu entwickeln. Der \hyperlink{OTP}{OpenTripPlanner} ist ein Programm, welches diesen Anforderungen entspricht. \newline

Während des Fachmoduls wurden wir auf den \hyperlink{CSA}{Connection Scan Algorithmus} aufmerksam. Dieser ist ein moderner Wegfindungsalgorithmus, welcher einen komplett anderen Ansatz als die bisherigen Wegfindungsalgorithmen wählt. Er sollte eine bessere Performance liefern als die \hyperlink{dij}{Dijkstra-basierten Algorithmen}. Er wird jedoch nicht oft verwendet, da er andere Voraussetzungen an die Datenstruktur und den Programmablauf hat als die Dijkstra-basierten Algorithmen. \newline

Ziel dieser Arbeit ist es nun, den Connection Scan Algorithmus für den OpenTripPlanner zu implementieren und dessen Performance zu testen. \newline

Zuerst werden die für das Projekt wichtigen Grundlagen erläutert. Anschliessend werden die im Projekt verwendeten Methoden erklärt. Dann wird der Connection Scan Algorithmus sowie die von ihm benötigte Datenstruktur genau erläutert. Als nächstes wird unser Programmierprozess genau dargestellt. Zum Schluss werden die Resultate des Performance-Tests präsentiert und analysiert.