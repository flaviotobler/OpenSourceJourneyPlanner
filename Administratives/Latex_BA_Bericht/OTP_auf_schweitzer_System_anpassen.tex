Der OpenTripPlanner konnte nicht ohne weiteres mit den GTFS- und OSM-Daten der Schweitz betrieben werden. 
Die GTFS-Daten der SBB verwenden den GTFS-Routetype "1700 Miscellaneous". Dieser ist offiziell in den GTFS-Standarts vorhanden,  [Verweis auf gtfs standarts] wird jedoch vom OTP nicht unterstützt, da der Typ Miscellaneous keinem Transportmittel zugewiesen werden kann und somit im Graphen nicht gewichtet werden kann. [Link auf meine Frage im der OTP Mailing List] Eine Analyse der Schweitzer GTFS-Daten ergab, dass der Typ für einen Teil der Seilbahnen sowie für die Autoverlade verwendet wird. Beide Typen sind für unsere Anwendung nicht von Bedeutung. Wir haben die Typerkennung umgeschrieben, so dass der RouteType 1700 nun als TraverseMode.Car to interpretiert wird. Der Typ Auto wird zwar unterstützt, jedoch nicht für die Routenberechnung der öffentlichen Verkehrsmittel verwendet. So können wir den Fehlerfall umgehen ohne den Algorithmus zu beeinflussen.
Die GTFS-Datem der SBB verwemdem ausserdem dem GTFS-Routetype "1500 Taxi" welcher nicht vom OTP unterstützt wird. Aus den zuvor genanten Gründen haben wir ihn auch auf den TraverseMode.Car rerouted.