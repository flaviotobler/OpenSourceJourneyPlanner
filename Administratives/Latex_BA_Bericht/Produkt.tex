\section{Produkt}



\subsection{Modellierung der Datenstrukur}
Der \hyperlink{timeTable}{\texttt{TimeTable}} muss alle Daten der Fusswege (\texttt{FootpathCSA}), Haltestellen (\texttt{StopCSA}), Fahrten (\texttt{TripCSA}), Verbindungen (\texttt{ConnectionCSA}) zur Verfügung stellen. Dies kann sie, indem wir Sets (Mengen) verwenden. Der Grund, wieso wir Sets verwenden, ist, dass es keine Duplikate geben kann, den ein Stop darf nur einmal in der Sammlung von \texttt{stops} vorkommen.
\begin{figure}[h]
	\centering
	\includegraphics[width=12cm]{img/Sammlungen.png}
	\caption{Der Zeittafel zur Verfügung stehende Sammlungen.}
	\label{fig:Sammlungen}
\end{figure}
\newline
Wie man in der Abbildung \ref{fig:Sammlungen} erkennen kann, verwenden wir HashSets für Haltestellen (\texttt{stops}), Fahrten (\texttt{trips}) und Fusswege (\texttt{footpaths}), weil diese Sammlung nicht sortiert sein muss. Im TimeTableBuilder verwendeten wir noch das TreeSet, um die Verbindungen (\texttt{ConnectionCSA}) zu sortieren bei ihrer Erzeugung. Beim TimeTable haben wir die gleiche Sammlung von Verbindungen, jedoch in LinkedHashSet erzeugt, weil wir zuvor versuchten, die Sammlungen zwischenzuspeichern mit Textfiles (Jackson 2). Da dies jedoch nicht mit TreeSets funktionierte, änderten wir diese zu LinkedHashSets. Der einzige Unterschied zwischen LinkedHashSet und HashSet ist, dass die Daten die Reihenfolge so beibehalten, wie man sie auch hinzugefügt hat. Da die Sammlungen (\texttt{connectionsAscending} und \texttt{connectionsDescending}) nachdem Erstellen nicht mehr ändern, benötigten wir also nicht zwingend ein TreeSet, denn sie können auch mit einem LinkedHashSet umgesetzt werden.


\subsection{Automatisch generierte Klassendiagramme}
Klassendiagramme können sehr hilfreich sein bei der Entwicklung. Sie geben einen Überblick, welche Klassen miteinander agieren.
Der OpenTripPlaner besteht aus unzähligen verschiedenen Klassen. Um eine bessere Übersicht zu erhalten, erzeugten wir mit dem Programm objectaid\footnote{\url{http://www.objectaid.com/home}} (ein Plugin für Eclipse) fast vollautomatisch Klassendiagramme. \newline

Dies stellte sich jedoch als Problem heraus, da der OpenTripPlanner viele Files mit mehreren Klassen enthalten. Zum Beispiel das File \texttt{GTFSRealtime} beinhaltet alleine schon 53 Klassen und 21000 Codezeilen.

\subsection{Dummy-GTFS-Daten erstellen}
Wir haben selbst \hyperlink{GTFS}{GTFS-Daten} erstellt, weil der TimeTableBuilder mit dem kompletten Schweizer-GTFS ein paar Stunden braucht, um die Daten einzulesen. So brauchen wir nicht bei jedem Ausführen des Programms jedes Mal ein paar Stunden zu warten, was uns bei der Entwicklung viel Zeit erspart. Indem wir diese GTFS-Daten selber erstellt haben, wissen wir, nun was genau vorhanden ist und können dadurch nachvollziehen, ob z.B. die GTFS-Daten richtig eingelesen wurden und können daraus auch weitere Methoden auf ihre Richtigkeit hin überprüfen.\newline

Das Dummy-GTFS wurde anhand der bestehenden Schweizer-GTFS-Daten erstellt. Im Schweizer-GTFS sind auch die Daten des öffentlichen Verkehrs von Liechtenstein enthalten. Wir haben uns entschieden, einen Teil der Liechtensteinischen Busverbindungen zu übernehmen und selber zu erstellen, um Daten zu verwenden, welche im Schweizer-GTFS auch vorhanden sind.

\begin{figure}[h]
	\centering
	\includegraphics[width=12cm]{img/LiniennetzDummyGTFS.png}
	\caption{Übersicht der Haltestellen und Routen im Dummy-GTFS.}
	\label{fig:DummyGTFS-uebersicht}
\end{figure}




\subsection{Mocking}


\subsubsection{CSAMock}
Das \hyperlink{mocking}{Mocking} des CSA bekommt ein Zeitplan-Objekt als Eingabe und speichert dieses zur Kontrolle in einem JSON-File. Danach wird manuell ein Journey erstellt welcher eine Reise von «Heerbrugg, Dornacherhof» nach «Heerbrugg, Bahnhof» repräsentiert. Diese Verbindung wurde gewählt, da es eine einfache Busfahrt ohne Zwischenstops ist. Dieser Journey wird dann als Antwort zurückgegeben.

\subsubsection{TimeTableBuilderMock}
Das Mocking des TimeTableBuilders erstellt manuell einen Zeitplan mit zwei Haltestellen und einer Busverbindung. Dieser wird anschliessend als Response zurückzugeben.

\subsubsection{JourneyToTripPlanConverterMock}
Der als Parameter bekommene Journey wird zur Überprüfung in einem JSON-File gespeichert. Anschliessend wird manuell eine Webseitenantwort aufgebaut. Anfangs wurde eine Reise ohne Zwischenstops, Umsteigen und Fusswege zurückgegeben. Doch in den folgenden Schritten wurde das Mocking erweitert, um die zuvor erwähnten, komplexeren Fälle zu behandeln.

\subsection{TimeTableBuilder}
Die Aufgabe des \hyperlink{timeTableBuilder}{TimeTableBuilder} ist es, Daten aus dem GTFS zu lesen und anschliessend eine Datenstruktur daraus zu erzeugen, die der Connection Scan Algorithm benötigt. Zum Einlesen der GTFS-Daten wird die Onebusaway Library verwendet. Diese erzeugt anhand der Daten Objekte. Die Objektdaten der Onebusaway Library sind in Listen abrufbar z.B. entspricht im GTFS Stop.txt ein Zeileneintrag einem Stop Objekt. Durch diese Library ist es einfacher, Daten aus dem GTFS abzufragen, z.B. kann man von einem Stop gezielt nach dem Namen fragen \texttt{stop.getName()}.\newline  

Durch diese Library lassen sich nun relativ einfach die Daten aus dem GTFS lesen und anschliessend mit den gewünschten Daten die benötigten Objekte erstellen. Dafür wird die Funktion \texttt{.loadFromGtfs()} ausgeführt.  Stop, Footpath und Trips lassen sich relativ leicht erstellen, weil die GTFS-Daten hierbei schon in sehr ähnlicher Form aufgebaut sind. Jedoch ist es bei der Connection etwas schwieriger, da die Daten nicht in dieser Art direkt vorhanden sind. Die Connections müssen bei Stop\_times.txt immer zwei hintereinanderfolgende Einträge (Objekte) vergleichen, um zu erkennen, ob es sich um eine Connection handelt. Eine Connection wird erkannt, wenn sie auf demselben Trip stattfinden, und die stop\_sequence Zahl muss grösser sein als der vorhergehende Eintrag. Nur wenn dieser Fall zutrifft, wurde eine Connection erkannt. Bevor jetzt aber das Connection Objekt einfach erzeugt werden kann, müssen noch die richtigen Referenzen gefunden werden, weil die Stops und Trips schon zuvor erzeugt wurden. Denn eine Connection muss wissen, auf welchem Trip sie ist und welche Departure- und ArrivalStops sie enthält. Nachdem also eine Connection gefunden wurde, wird der richtige Trip in der bestehenden Liste gesucht: Anhand der TripId kann der Trip aus der Liste gefunden werden. Dasselbe geschieht auch mit den Departure- und ArrivalStops. Wenn alle Informationen gefunden wurden, wird das Connection Objekt erzeugt und in das TreeSet (sortiere Liste) übergeben. Durch die \texttt{.compareTo()} Funktion der Connection lässt es sich an der richtigen Stelle im TreeSet einfügen. Somit bekommen wir alle Connections sortiert nach der Departuretime. Am Ende der Funktion wird ein Java-SerializedObjectFile vom TimeTable erstellt. Somit wird das gewünschte Format zwischengespeichert. Das Erstellen dieser Datei benötigt nur 1-2 Minuten, und die Datei ist 367MB gross.\newline

Bevor wir uns entschlossen haben, das Java-SerializedObjectFile am Ende der Funktion zu erstellen, versuchten wir es stattdessen mit Textfiles, die durch Jackson 2 erzeugt wurden. Jackson 2 ist eine Library zum Konvertieren von JSON (Textfile) zu Java-Objekten und umgekehrt. Doch Jackson 2 hatte ein grosses Problem. Die Files wurden zwar wieder mit den Angaben richtig eingelesen, jedoch waren die Referenzen auf die richtigen Objekte nicht vorhanden, z.B. ist ein Stop einzigartig und sollte nur einmal vorkommen, jedoch wurden dann mehrere Stops erzeugt mit denselben Namen und Werten. Dies führte dann wiederum zu Problemen beim Algorithmus, Jackson 2 konnte deshalb nicht verwendet werden. \newline

Die Funktion \texttt{.loadFromGtfs()} benötigt etwa 60h, um das Schweizer GTFS einzulesen., aber durch Optimierung konnte die Zeit verringert werden auf 22h indem wir nicht mehr einfach ständig wieder nach dem Trip suchten wie vorher, wenn eine Connection gefunden wurde. Durch Zwischenspeichern der Referenz des vorhergehenden Trips kann dieser wiederverwendet werden, wenn die TripId der Einträge übereinstimmen. Somit muss nur ein Trip in der Liste gesucht werden, wenn diese TripId nicht übereinstimmen. Zudem kann durch Zwischenspeichern der Referenz auf den vorhergehenden Stop eine Suche in der Liste vermieden werden. Da ein Trip und Stop nur einmal vorkommt in der Liste, kann beim Durchlaufen mit einer Suchschleife die Suche nach dem Trip vorzeitig abgebrochen werden, was verhindert, dass unnötig weiter nachdem Trip/Stop gesucht wird.  \newline

Die Funktion \texttt{.loadFromSerializedObjectFile()} kann verwendet werden, um zwischengespeicherte TimeTables einzulesen. Das SerializedObjectFile, das zuvor aus dem Schweizer GTFS erstellt wurde, lässt sich innerhalb von 151s einlesen.

\subsection{JourneyToTripPlanConverter}
Der \hyperlink{journeyToTripPlanConverter}{JourneyToTripPlanConverter} bildet einen von der Webseite benötigten TripPlan aus den vom CSA generierten Journeys. Aus dem Mocking sind die benötigten Parameter schon bekannt, so dass die Generierung nun nur noch automatisiert werden muss. Dabei gab es jedoch mehrere Punkte, welche speziell beachtet werden mussten.
\begin{enumerate}
	\item Ein Journey besitzt nur eine Dauer für den kompletten Weg. Der TripPlan jedoch speichert sich separate Werte für Fahrzeit, Laufzeit und Wartezeit. Die Fahrzeit und die Laufzeit können dabei einfach aus den Start- und Stop-Zeiten der Legs oder Footpaths übernommen werden. Bei der Wartezeit besteht jedoch das Problem, dass sie den Zeitunterschied zwischen zwei Abschnitten repräsentiert. Dies bedeutet, dass in der Schleife die beiden zu vergleichenden Zeiten nicht gleichzeitig vorhanden sind. Dieses Problem wird umgangen, indem die Endzeit des Itinerary nach jedem berechneten Leg angepasst wird. Somit stimmt dieser Wert immer mit der Endzeit des Leg des vorhergehenden Schleifendurchgangs überein. 
	\item Jedes Leg im TripPlan benötigt ein LegGeometry-Object. Dies wird benötigt, damit die Webseite eine Linie entlang des Fahrtweges anzeigen kann. Dabei wird mithilfe der GeometryUtils-Bibliothek und der Start- und Endkoordinaten eine Gerade erstellt. Diese Koordinaten werden dann zusammengesetzt und bilden den Reiseweg ab.
	\item Im TripPlan wird ein Fussweg mithilfe von verschiedenen WalkSteps definiert. Diese benötigen jedoch eine Variable AbsoluteDirection, welche acht Himmelsrichtungen repräsentiert. Dazu werden die Koordinaten mithilfe einer Kompass-Funktion in einen Winkel umgewandelt. Danach wird der Winkel auf 45 Grad-Abschnitte gerundet und den jeweiligen Himmelsrichtungen zugeordnet.
	\item Der CSA stellt sowohl Fusswege als auch Umsteigeprozesse als Footpaths dar. Umsteigeprozesse werden aber vom TripPlan nicht dargestellt, ausser dass sie in die Zeitberechnung miteinfliessen müssen. Daher müssen die Fusswege und Umsteigeprozesse unterschieden werden. Nachdem ein Leg generiert wurde, jedoch noch bevor es der Liste von Legs hinzugefügt wird, wird überprüft, ob die Start- und Zielkoordianten gleich sind. Ist dies der Fall, so handelt es sich um einen Umsteigeprozess und das berechnete Leg wird der Liste nicht hinzugefügt. Die berechneten Zeiten werden jedoch trotzdem für das nächste Leg verwendet.
\end{enumerate}

\subsection{ConnectionScanAlgorithm}
Von den beiden \hyperlink{CSA}{CSA-Versionen}\cite{csa} kümmerten wir uns zuerst um den \hyperlink{EAS}{Earliest Arrival Scan Algorithmus (EAS)}, da dieser weniger komplex ist und als Grundlage für den \hyperlink{PCS}{Profile Connection Scan Algorithmus (PCS)} dient.
\subsubsection{EAS}
Am Anfang werden für alle Stops und Trips Handler angelegt. Handler sind Hilfskonstruktionen, welche alle Informationen der Stops und Trips beinhalten, welche im Verlauf der Berechnung geändert werden müssen wie zum Beispiel die EinstiegsConnection für die Trips. Das ist nötig, da wir mit einem persistenten TimeTable-Objekt arbeiten, welches für die nächsten Anfragen nicht verändert werden darf.
\newline


Danach werden die Zeitvariablen vorbereitet. Das Jahr, der Monat und der Tag müssen aus dem Request übernommen werden, da sich die Zeitangaben des Zeitplans nur auf die Uhrzeit und nicht auf das Datum beziehen. Dazu werden die Datumswerte in Variablen gespeichert.  Nun wird im Stop-Handler des Startstops die Startzeit des Requests eingetragen. Für alle anderen Stop-Handler wird die Zeit auf den 31.12.20000 gesetzt. Diese Zeit simuliert eine unendlich grosse Zahl, welche trotzdem noch mit den Calendar-Methoden verglichen werden kann.
\newline


Dann werden alle aufsteigend nach Abfahrtszeit sortierten Connections durchlaufen. Für jede Connection wird überprüft, ob die im Stop-Handler des Startstops der Connection gespeicherte Zeit vor der Abfahrtszeit der Connection ist oder ob im zur Connection passenden Trip-Handler das Trip-Bit gesetzt ist. Das Trip-Bit ist am Start auf «false» gesetzt. Sobald eine Connection gefunden wird, welche eine der zwei vorherigen Bedingungen erfüllt (vgl. Kap. 7.6), so wird es für den zur Connection passenden Trip auf True gesetzt. Damit werden erreichbare Verkehrsmittel vermerkt, so dass weitere Connections im gleichen ÖV auch als erreichbar markiert sind. Die zweite Bedingung prüft, ob der Startstop der Connection schon erreicht wurde. Anfangs ist nur die Zeit im Stop-Handler des Startstops gesetzt. Alle anderen Zeiten sind auf unendlich gesetzt. Somit werden nur Connections behandelt, welche vom Startstop ausgehen und später abfahren als die im Request definierte Startzeit. Sobald eine solche Connection gefunden wurde, wird die Ankunftszeit in den Stop-Handler des Ankunftsstops der Connection geschrieben. Somit ist nun auch dieser als erreichbar markiert und wird bei weiteren Durchläufen beachtet. Zusätzlich wird jedes Mal, wenn eine Connection gefunden wurde, ein JourneyPointer im Stophandler des Ankunftsstops der Connection gespeichert, um den Journey später rekonstruieren zu können.
\newline


Das Break-Kriterium für die Schleife zeigt sich, wenn die Abfahrtszeit der Connection später ist als die im Stop-Handler des Zielstops gespeicherte Zeit. Diese Zeit ist auf unendlich gesetzt und wird erst neu gesetzt, wenn ein Journey zum Zielpunkt gefunden wurde. Da die Connections aufsteigend nach Abfahrtszeit durchlaufen werden, ist dies auch automatisch die am frühesten endende Reise. 
\newline


Nun wird vom Zielstop aus der Journey rekonstruiert. Im JourneyPointer des Zielstops steht der zuvor erreichte Stop. Nun wird der JourneyPointer dieses neuen Stops untersucht. Dies wird so lange wiederholt, bis der Startstop erreicht ist. Nun werden die gefundenen Stops in umgekehrter Reihenfolge in den Journey eingetragen und der Journey wird zurückgegeben.
\subsubsection{PCS}
Der \hyperlink{PCS}{Profile Connection Scan Algorithmus (PCS)} erstellt am Anfang auch Stop- und Triphandler sowie die im EAS erwähnten Zeitvariablen. Danach werden die Verbindungen in einer Schleife durchlaufen. Im Gegensatz zum EAS werden sie jedoch absteigend nach Abfahrtszeit durchlaufen. Damit jedoch nur ein Zeitfenster der Verbindungen um die gesuchte Zeit herum behandelt werden muss, wird vor der Schleife ein EAS durchgeführt, welcher nur die früheste Ankunftszeit zurückliefert. Dieser Zeit werden dann zwei Stunden hinzugerechnet und sie wird als Einstiegspunkt für die Schleife benutzt. 
\newline


Nun werden drei Zeitvariablen gesetzt. Die erste zeigt, wann und ob man ans Ziel kommt, wenn man aus dem ÖV aussteigt. Dies ist nur der Fall, wenn der Ankunftsort der Connection dem Zielort entspricht. In diesem Fall wird die Ankunftszeit der Connection in der Zeitvariable gespeichert. Ist dies nicht der Fall, so wird sie auf unendlich gesetzt.
\newline


Die zweite Zeitvariable zeigt, wann und ob man ans Ziel kommt, wenn man im ÖV sitzen bleibt. Dabei wird die TripZeit in der Zeitvariablen gespeichert. Ist diese ungleich unendlich, so ist das Ziel mit Weiterfahren erreichbar.
\newline


Die dritte Zeitvariable zeigt, wann und ob man ans Ziel kommt, wenn man in ein anderes ÖV umsteigt. Dazu werden alle TimeTupels im Zielstop aufsteigend durchlaufen, bis eine TimeTupel gefunden wurde, welche später abfährt als die Ankunftszeit der Connection plus die Umsteigezeit. Da jeder Stop ein default TimeTupel mit der Zeit unendlich hat, wird immer eine Möglichkeit gefunden. 
Danach werden die drei Zeitvariablen verglichen und die früheste Zeitvariable wird behalten. Dies ist nun die schnellste Zeit, in welcher man von dieser Connection aus das Ziel erreichen kann. Nun wird von der Abfahrtszeit der Connection die Umsteigzeit abgezogen, und es wird mit diesen beiden Zeiten ein neues TimeTupel erstellt. Nun wird überprüft, ob die Ankunftszeit nicht unendlich ist und somit der Zielort erreichbar ist. Wenn dies der Fall ist, so wird das Tupel zusammen mit einen JourneyPointer in den Stop-Handler geschrieben, für den Trip wird eine TripZeit festgelegt, und falls für diesen Trip noch keine ExitConnection festgelegt ist, wird die aktuelle Connection eingetragen. Dies wird so lange wiederholt, bis die Abfahrtszeit der Connection vor der im Request spezifizierten Startzeit ist.
\newline


Nun werden die Journeys vom Startstop aus rekonstruiert. Es wird ein Journey für jedes im Startstop gespeicherte TimeTupel generiert. Dabei wird der erste JourneyPointer im Journey eingetragen. Dann werden die im Endstop des Journey gespeicherten TimeTupels überprüft. Hierbei wird nur das erste Tupel und nicht alle verwendet. Diese Entscheidung wurde aus Performance-Gründen getroffen, da die Anzahl an Schritten ansonsten exponenziell ansteigen würde. Dies wird so lange wiederholt, bis der Zielstop erreicht ist. 
\newline


Der Vorgang benötigt jedoch noch einen Zusatz, da der Algorithmus auch Reisen findet, welche Kreise fahren und denselben Ort mehrfach anfahren. Um dies zu verhindern, wird eine Liste mit allen schon angefahrenen Stops geführt. Für jeden neuen JourneyPointer wird geprüft, ob der Ankunftsstop schon in der Liste vorhanden ist. Ist dies der Fall, so wird ein Bit auf TRUE gesetzt. Am Ende werden nur jene Journey der Rückgabe hinzugefügt, bei welchen kein Stop mehrfach angefahren wurde.

\subsection{Performance-Test}
Um den Performance-Test durchzuführen, wird ein JUNIT-Test mit der JUNIT-Benchmark-Erweiterung genutzt. Als Grundlage der Tests dient der EAS \hyperlink{EAS}{(vgl. Kap. 6.2.4)}. Mit diesem wurden mehrere Tests durchgeführt, welche den Flaschenhals des Algorithmus aufzeigen sollten.
\newline


Ein Request auf dem gesamtschweizerischen Netzwerk mit dem EAS dauerte ca. 24 Stunden. Um die Dauer einer Anfrage auf eine annehmbare Zeit zu minimieren, so dass wir genauere Analysen durchführen konnten, verwendeten wir ein etwas kleineres GTFS, nämlich das von Portland. Dabei stellten wir fest, dass unser hauptsächliches Break-Kriterium aufgrund von einigen Methoden, welche wegen der JUnit-Benchmark zusätzlich als static definiert werden mussten, nicht mehr funktionierte. Das Break-Kriterium soll in Kraft treten, wenn der Algorithmus eine Lösung gefunden hat. Dies wurde aber nicht mehr ausgelöst, somit lief er einfach durch alle Connections, ohne einen Treffer zu landen. Somit waren die im Test erzeugten Zeiten nicht repräsentativ für unseren Algorithmus. \newline

Alternativ haben wir die Funktion \texttt{System.currentTimeMillis()} verwendet. Diese Funktion gibt die aktuelle Zeit in Millisekunden zurück. Durch diese Funktion ist es also möglich, sie zu stoppen für beliebige Abschnitte, indem man am einem gewählten Anfangspunkt/Endpunkt die Zeit speichert. Durch die Differenz der zwei Zeiten von Anfangs- und Endpunkt lässt sich nun die Zeitspanne berechnen, die das Programm benötigt hat. 

\subsection{OTP mit Schweizer Daten implementieren}
Der originale OTP muss mit den Schweizer GTFS-Daten funktionieren, da er als Grundlage für den Algorithmus dient und die Performance der beiden Algorithmen verglichen wird. Es gab zwei Probleme, welche dabei behandelt werden mussten.
\newline

Die SBB beschreiben in ihren GTFS-Daten einige Verbindungen mit dem Routentyp 1700 «Miscelaneous». Dieser Typ wird jedoch vom OTP nicht unterstützt, da er nicht zugeordnet werden kann. Nach einer Überprüfung der SBB-Daten stellte sich heraus, dass der Routentyp 1700 nur für einige Sessellifte sowie Autoverladestationen verwendet wird. Da diese beide nicht von unserem Projekt behandelt werden, können sie ignoriert werden. Dazu fügten wir dem OTP einen Handler für den Routentyp 1700 hinzu, welcher ihn als Auto definiert. Da das Auto kein öffentliches Verkehrsmittel ist, ist es für unser Projekt nicht relevant, es wird jedoch vom OTP unterstützt.
\newline


Das zweite Problem war der benötigte Arbeitsspeicher. Für die Graphberechnung mit den SBB-GTFS-Daten benötigt der OTP ca. 15 GB Arbeitsspeicher. Die von uns benutzten Rechner konnten diese Speichermenge jedoch nicht zur Verfügung stellen. Deshalb wird für die Berechnung mit den grossen Datensätzen ein Laborcomputer der HTW verwendet. Dieser entspricht mit seinem 32GB-Arbeitsspeicher den Anforderungen.


\subsection{Kann OTP ohne .osm file ausgeführt werden?}
Das .osm steht für OpenStreetMap und ist eine Plattform\footnote{\url{https://www.openstreetmap.org/}}, die frei nutzbare Daten (sprich Geodaten) zur Verfügung stellt, wie z.B. Strassen, Wege, Gebäude, Haltestellen usw. OTP kann ohne .osm file ausgeführt werden. Das Programm erkennt automatisch, ob ein OSM-File vorhanden ist. Jedoch werden dann keine Fusswege mit einberechnet, um zu einer Haltestelle zu gelangen, wie man in Abbildung \ref{fig:ohneosmfile} erkennen kann.
Aber wenn OTP mit dem .osm File ausgeführt wird, lässt sich anhand der Informationen ein Gehweg zu der Haltestelle finden (inklusive Laufzeit und Distanz), wie man in Abbildung \ref{fig:mitosmfile} erkennen kann. 

\begin{figure}[htb]
	\centering
	\begin{minipage}{0.45\linewidth}
		\centering
		\includegraphics[width=6cm]{img/ohneosmfile.png}
		\caption{ohne .osm File}
		\label{fig:ohneosmfile}
	\end{minipage}
	%\hfill
	\begin{minipage}{0.45\linewidth}
		\centering
		\includegraphics[width=6cm]{img/mitosmfile.png}
		\caption{mit .osm File}
		\label{fig:mitosmfile}
	\end{minipage}
\end{figure}





