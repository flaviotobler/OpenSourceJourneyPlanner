\section{Resultate}
Die \hyperlink{performanceTest}{Tests} mit den gesamtschweizerischen \hyperlink{GTFS}{GTFS-Daten} mit der \gls{JUnit}-Benchmark führten zu folgenden Ergebnissen: Ein Request benötigt ca. 24h, um bearbeitet zu werden. Somit ist der Algorithmus um den Faktor 10'000 langsamer als der Originalalgorithmus des \hyperlink{OTP}{OTP}, welcher für dieselbe Anfrage nur wenige Sekunden benötigt. \newline

Die Tests mit den Portland-GTFS-Daten mit der Funktion \texttt{System.currentTimeMillis()} führten zu folgenden Ergebnissen: Ein Request kann in einem Zeitfenster zwischen 1 - 9 Minuten bearbeitet werden. Diese Zeitspanne hängt davon ab, nach wie vielen Schleifendurchläufen eine Lösung gefunden wurde und somit das Break-Kriterium greift. Damit ist der Request um den Faktor 1'000 langsamer als der Originalalgorithmus. \newline

Für einen Schleifendurchlauf gibt es zwei mögliche Zeiten. Ist die in der Schleife behandelte Connection nicht relevant, so wird sie nicht bearbeitet. Dabei wird eine Zeit von <1ms benötigt. Wenn die Connection jedoch relevant ist, wird eine Zeit von 15ms benötigt.
