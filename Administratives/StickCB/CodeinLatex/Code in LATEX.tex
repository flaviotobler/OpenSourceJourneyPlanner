\documentclass[a4paper,12pt]{scrartcl}

\usepackage[utf8]{inputenc}
\usepackage[ngerman]{babel}
\usepackage{cite} 
\usepackage{graphicx}
\usepackage[colorlinks=true, urlcolor=black, linkcolor=black, citecolor=black]{hyperref}
\graphicspath{{img/}}

\usepackage[]{algorithm2e}
\usepackage{listings}
\usepackage{color}

\definecolor{mygreen}{rgb}{0,0.6,0}
\definecolor{mygray}{rgb}{0.5,0.5,0.5}
\definecolor{mymauve}{rgb}{0.58,0,0.82}

\lstset{ 
	backgroundcolor=\color{white},   % choose the background color; you must add \usepackage{color} or \usepackage{xcolor}; should come as last argument
	basicstyle=\footnotesize,        % the size of the fonts that are used for the code
	breakatwhitespace=false,         % sets if automatic breaks should only happen at whitespace
	breaklines=true,                 % sets automatic line breaking
	captionpos=b,                    % sets the caption-position to bottom
	commentstyle=\color{mygreen},    % comment style
	deletekeywords={...},            % if you want to delete keywords from the given language
	escapeinside={\%*}{*)},          % if you want to add LaTeX within your code
	extendedchars=true,              % lets you use non-ASCII characters; for 8-bits encodings only, does not work with UTF-8
	frame=single,	                   % adds a frame around the code
	keepspaces=true,                 % keeps spaces in text, useful for keeping indentation of code (possibly needs columns=flexible)
	keywordstyle=\color{blue},       % keyword style
	language=Octave,                 % the language of the code
	morekeywords={*,...},            % if you want to add more keywords to the set
	numbers=left,                    % where to put the line-numbers; possible values are (none, left, right)
	numbersep=5pt,                   % how far the line-numbers are from the code
	numberstyle=\tiny\color{mygray}, % the style that is used for the line-numbers
	rulecolor=\color{black},         % if not set, the frame-color may be changed on line-breaks within not-black text (e.g. comments (green here))
	showspaces=false,                % show spaces everywhere adding particular underscores; it overrides 'showstringspaces'
	showstringspaces=false,          % underline spaces within strings only
	showtabs=false,                  % show tabs within strings adding particular underscores
	stepnumber=2,                    % the step between two line-numbers. If it's 1, each line will be numbered
	stringstyle=\color{mymauve},     % string literal style
	tabsize=2,	                   % sets default tabsize to 2 spaces
	title=\lstname                   % show the filename of files included with \lstinputlisting; also try caption instead of title
}






\begin{document}
	\pagenumbering{Roman}
	\newpage
	

\section{Source Code}
\label{sec:Source Code}

%Langer Code
\lstinputlisting[language=Java,label=Code1,caption=code1 zu sehen, firstline=1, lastline=221]{src/OTPMain.java}
%\lstinputlisting[frame=single,label=beispielcode,caption=Ein Beispiel]{beispiel.pl}



%Kurze Code
\begin{lstlisting}[caption={ein paar Zeilen code}\label{lst:test123},captionpos=b] 
for i:=maxint to 0 do 
begin 
j:=square(root(i)); 
end; 
\end{lstlisting}


%wenig moeglichkeiten nicht verwenden
\begin{verbatim}
hier kommt dann der Quellcode...
\end{verbatim}

\lstinline[]
{hier kommt der Quellcode...}






%https://en.wikibooks.org/wiki/LaTeX/Source_Code_Listings
%http://www.macwrench.de/wiki/Kurztipp_-_Quellcodelistings_in_LaTeX
%http://robert-kummer.de/2006/04/18/latex-quellcode-listing/
	
	

\section{Pseudo Code}
\label{sec:Pseudo Code}

\begin{algorithm}[H]
	\KwData{this text}
	\KwResult{how to write algorithm with \LaTeX2e }
	initialization\;
	\While{not at end of this document}{
		read current\;
		\eIf{understand}{
			go to next section\;
			current section becomes this one\;
		}{
			go back to the beginning of current section\;
		}
	}
	\caption{How to write algorithms}
\end{algorithm}

%https://en.wikibooks.org/wiki/LaTeX/Algorithms

	
	%\newpage
	%\listoffigures
	%	\bibliography{test2,Programme}
	\bibliographystyle{IEEEtran}
	
	
	%Verzeichnis für Listenings
	%\lstlistoflistings
	

	
	
	
	
	
\end{document}