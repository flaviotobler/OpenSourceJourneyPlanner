\documentclass[a4paper,12pt]{scrartcl}
\usepackage[left= 2.5cm,right = 2cm, bottom = 4 cm, top = 2.5cm]{geometry}
\usepackage[utf8x]{inputenc}
\usepackage {graphicx}
\usepackage[ngerman]{babel}
\usepackage{hyperref}


\begin{document}
\section{Einführung}
\label{sec:Einfuehrung}
In Zeiten der Digitalisierung des Öffentlichen Verkehrs wird umso mehr Software benötigt die gewisse Bedürfnisse befriedigt. Seit 2016 stellt die Plattform\footnote{\url{https://opentransportdata.swiss/}} aktuelle Datensätze für den Öffentlichen Verkehr der Schweiz zur Verfügung. Die Daten umfassen Fahrplandaten, Echtzeitdaten, Statistische Daten und noch mehr. Daraus lassen sich Programme realisieren wie z.B. ein Journey Planner der Verbindungsvorschläge inkl.Umsteigvorgänge erstellt. 

\subsection{Aufgabenstellung}
\label{Aufgabenstellung}




\subsection{Vorgehen}
\label{Vorgehen}

\subsection{Ziel der Bachelorarbeit}
\label{Ziel der Bachelorarbeit}


\end{document}